\chapter{The Bayesian Approach to Confidence Interval Construction for Population Attributable Risk (PAR)} \label{bayesian-model}
In this chapter I will go over the approach proposed in paper \textit{Bayesian methods of confidence interval construction for the population attributable risk from cross-sectional studies} by Pirikahu\&al. 

\section{Mathematical Model} \label{PiriMath}
\begin{table}[h!]
\centering
\caption{2 x 2 Contingency Table For n Samples}
\label{contingency-table}
\begin{tabular}{|c|c|c|c|}
\hline
Exposed & D+ (has disease) & D- (no disease) & Total \\ \hline
E+ & a & b & a $+$ b \\ \hline
E- & c & d & c $+$ d \\ \hline
Total & a $+$ c & b $+$ d & n \\ \hline
\end{tabular}
\end{table}

$n$ denotes the total sample size where $a + b + c + d = n$. From the contingecy table we can see that a cross-sectional study with one exposure variable and one desease variable can be seen as a multinomial distribution with four possible outcomes that are independet from each other. These outcomes can be described with multinomial distribution as follows:
\begin{equation} \label{multinomial}
(a, b, c, d) \sim Multinomial(n, p_{11}, p_{12}, p_{21}, p_{22})
\end{equation}

\begin{itemize}
    \item $P(E+) = \frac{a + b}{n}$: The propability of the expososed in a population.
    \item $P(E-) = \frac{c + d}{n}$: The propability of the expososed in a population.
    \item $P(D+) = \frac{a + c}{n}$, The propability of having the disease in a population.
    \item $p_{11} = P(D + \cap E+) = P(D + | E+)P(E+) = P(E + | D+)P(D+) = \frac{a}{n}$: The probability of being exposed and not having the disease.
    \item $p_{12} = P(D - \cap E+) = P(D - | E+)P(E+) = P(E + | D-)P(D+) = \frac{b}{n}$: The probability of being exposed and not having the disease.
    \item $p_{21} = P(D + \cap E-) = P(D + | E-)P(E-) = P(E - | D+)P(D+) = \frac{c}{n}$: The probability of not being exposed and having the disease.
    \item $p_{22} = P(D - \cap E-) = P(D - | E-)P(E-) = P(E - | D-)P(D-) = \frac{d}{n}$: The probability of not being exposed and not having the
\end{itemize}

The population attributable risk (PAR) is defined as the proportion of the disease in a population that have occured due to exposure. The PAR can be calculated with formula \ref{PARequation}. With Bayes' theorem we can rewrite the formula and by add the values from the list above we get the maximum likehood estimation function for PAR.

\begin{equation}
\begin{split}
PAR &= P(D+) - P(D+| E-) \\
    &= P(D+) - \frac{P(E-|D+)P(D+)}{P(E-)} \\
    &= \frac{a + c}{n} - \frac{\frac{c}{n}}{\frac{c + d}{n}} \\
    &= \frac{a + c}{n} - \frac{n}{c} \times  \frac{c + d}{n} \\
    &= \frac{a + c}{n} - \frac{c + d}{c} \\
    &= \frac{a + c}{a + b + c + d} - \frac{c + d}{c} \\
\end{split}
\end{equation}

Prior distribution that estimates all the situations dercribed in the contingency table is: 
$\theta = (p_{11}, p_{12}, p_{21}, p_{22})$. Observed values or samples are: $x = (a, b, c, d)$. And the likelihood in respect to $p_k$ is denoted by a propability mass function: 
\begin{equation}
    f(x|\theta) = \frac{n!}{a!b!c!d!}p_{11}^ap_{12}^bp_{21}^cp_{22}^d
\end{equation}

Posterior distribution is: $p(a,b,c,d|p_{11}, p_{12},p_{21},p_{22}) = p(\theta|x) \varpropto  f (x|\theta)p(\theta)$. Due to conjugacy relationship, in relation to the prior, posterior can be found analytically. Posterior is:
\begin{equation}
    \theta|x ~ Dirichlet(a + 1, b + 1, c + 1, d + 1).
\end{equation}
It is computationally much less expencive to represent posteriors analytically than with MCMC simulation. 

Confidence interval is a frequentist consept but it can be constructed based on the repeated sampling of the posterior even with this Bayesian approach. \cite{Pirikahu2016BayesianMO}

\section{Code Implementation} \label{CodeImplementation}
\begin{lstlisting}[language=R]
def hello_world():
    print("Hello, world!")
\end{lstlisting}