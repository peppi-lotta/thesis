\chapter{Evaluation} \label{sec:Evaluation}

To explore performance we will select known values for parameters p, q, e and n. 
\begin{itemize}
    \item $p = P(D^+ |E^+)$
    \item $q = P(D^+ |E^-)$
    \item $e = P(E^+)$
    \item $n$
\end{itemize}

We need to generate 10,000 contingency tables for these chosen variables based on the multinomial model \ref{multinomial}. A confidence interval will be constructed for the PAR with two different methods; bootstrap and the Bayesian approach proposed by Pirikahu et al.

Parameter values for the simulation are:
\begin{table}[h!]
    \centering
    \caption{Parameters for the simulation}
    \label{sample-parameters}
    \begin{tabular}{|c|c|c|c|c|c|c|c|c|c|}
    \hline
    $p$ & 0.01 & 0.05 & 0.1 & 0.2 & 0.3 & 0.35 & 0.4 & 0.45 & 0.5 \\ \hline
    $q$ & 0.001 & 0.05 & 0.1 & 0.2 & 0.3 & 0.35 & 0.4 & 0.45 & 0.5 \\ \hline
    $e$ & 0.1 & 0.2 & 0.3 & 0.4 & 0.5 & 0.6 & 0.7 & 0.8 & 0.9 \\ \hline
    \end{tabular}
    \end{table}

\section{Comparison of Pirikahu et al.'s Bayesian method with standard frequentist approaches.}

\subsection{Small datasets}

\subsection{Priors}
The tail-based Bayesian confidence intervals at Jeffreys prior performs
best in terms of mean coverage probability. \cite{Shi2009BayesianCI} 
